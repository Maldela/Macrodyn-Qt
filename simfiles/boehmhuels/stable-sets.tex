\subsection{Approximation of stable and unstable sets}
Stable and unstable sets of the fixed point $\xi$ give important
information on the underlying dynamics. These sets are defined as 
\begin{eqnarray*}
W^s(\xi) &=& \{ p \in \R^2: \lim_{t \to \infty}\|F^t(p) - \xi\| = 0\},\\
W^u(\xi) &=& \{ p \in \R^2: \exists (p_t)_{t\in\Z_-}: p_{t} =
F(p_{t-1}),\ p_0 = p,\ \lim_{t \to -\infty}\|p_t - \xi\| = 0\}.   
\end{eqnarray*}
Here $F^t$ denotes the $t$-th iteration of the map $F$.

For computing these objects numerically we first observe that one
immediately gets an approximation of the unstable set, by iterating
points from 
the unstable subspace of the fixed point $\xi$ in forward time. Due to
the lack of an inverse, we cannot apply the same approach for
computing the stable set. However, the following proposition shows an
alternative. 

\begin{proposition}\label{prop1}
Let $\delta \in (0,1)$.
For the map $F$, defined in \eqref{final}, the set
$$
S_0 = \left\{
  \begin{pmatrix}
   0\\y 
  \end{pmatrix}
  : y \in \R \right\}
$$
is $F$-invariant in forward time,
and part of the stable set of the fixed point $\xi$.

Furthermore, the following sets are also part of the stable set:
$$
S_t = \left\{p \in \R^2: (F^{t-1}(p))_1 \neq 0 \text{ and } (F^t(p))_1
  = 0\right\},\quad t = 1,2,\dots, 
$$
where $(F^t(p))_1$ denotes the first component of $F^t(p)$.
\end{proposition}

\begin{proof}
For $y \in \R$ we get
$$
F^t
\begin{pmatrix}
0\\y  
\end{pmatrix}
= 
\begin{pmatrix}
0\\\delta^t y  
\end{pmatrix}
\to \xi \ \text{as}\ t \to \infty. 
$$
Consequently, $S_0$ is invariant and part of the stable set.

Let $p\in S_t$ for a $t \ge 1$, $t\in \N$. Then $F(p)\in S_{t-1}$
and thus, $F^{t}(p) \in S_0$, which proves that $p$ lies in the stable
set.
 
\end{proof}

For computing these sets explicitly, we observe that 
$$
\left( F
\begin{pmatrix}
x \\ y  
\end{pmatrix}
\right)_1 = 0
\Leftrightarrow
x = 0 \quad \text{or}\quad y = \alpha+\frac {1-\delta}{\delta},
$$
therefore $S_1=\left\{
  \begin{pmatrix}
  x  \\ \alpha+\frac {1-\delta}{\delta} 
  \end{pmatrix}: x \in \R\setminus 0 \right\}$.

Let $x_t = (F^t(x,y))_1$,  $y_t = (F^t(x,y))_2$, $t\in \N$, then
we can iterate this process further and obtain
$$
x_2 = x_1\left(\delta +\frac {1-\delta}{\alpha - y_1}\right) 
= x_0\left(\delta +\frac {1-\delta}{\alpha - y_0}\right)
\left(\delta +\frac {1-\delta}{\alpha - y_1}\right).
$$
The zeros of the last factor (that are no zeros of the first terms) 
define $S_2$. 
The computation of $S_t$ requires to solve  $\left(\delta +\frac
  {1-\delta}{\alpha - y_{t-1}}\right) = 0$ which is rather involved,
since one has to replace $y_{t-1}$ iteratively by $y_0$ which gives
highly nonlinear equations. Figure \ref{F2} shows the results of the
corresponding algorithm.

\begin{figure}[H]
\begin{center}
\psfrag{S0}{$S_0$}
\psfrag{S1}{$S_1$}
\psfrag{S2}{$S_2$}
\psfrag{S3}{$S_3$}
\psfrag{S4}{$S_4$}
\psfrag{S5}{$S_5$}
\psfrag{S6}{$S_6$}
\psfrag{x}{$x$}
\psfrag{y}{$y$}
\epsfig{width = 14cm, file=bilder/stab.eps}  
\end{center}
\caption{Stable sets $S_t$ for $t=0,\dots,8$ (left) and $t=0,\dots,40$
  (right) for the parameters $\delta = 0.8$, $\alpha = -0.04$.\label{F2}}
\end{figure}

\subsection{Homoclinic orbits}
The plot (of parts) of the stable and unstable set in Figure \ref{F3} exhibits
transversal intersections at the parameter $\delta = 0.8$,
$\alpha = -0.03$. Points $p \in
W^s(\xi)\cap W^u(\xi)$ are called homoclinic and a corresponding 
trajectory $(p_t)_{t\in \Z}$ that satisfies $p_0 = p$ and 
$\lim_{t\to \pm \infty} \|p_t - \xi\| = 0$ is called a homoclinic orbit.
Numerically, we find these objects by solving boundary value
problems, cf.\ \cite{bk97b, bhkz04}.

\begin{figure}[H]
\begin{center}
\psfrag{x}{$x$}
\psfrag{y}{$y$}
\psfrag{xi}{$\xi$}
\epsfig{width = 10cm, file=bilder/hom1.eps}  
\end{center}
\caption{Approximation of stable (green) and unstable (red) sets of the
  fixed point $\xi$ and a homoclinic orbit for 
  $\delta = 0.8$, $\alpha = -0.03$. On the blue line $LC_{-1}$
  the Jacobian $DF(p)$ is non-invertible.\label{F3}}
\end{figure}

It is well known from the celebrated
Smale-{\v{S}}il'nikov-Birkhoff-Theorem that the dynamics in a
neighborhood of a homoclinic orbit is chaotic, cf.\ \cite{sm67, sh67}.
In case the dynamical system is invertible, the Shadowing Lemma, cf.\
\cite{py99, p00}, guarantees the existence of 
multi-humped orbits with short and long moderate phases, see Figure
\ref{F4}. 

\begin{figure}[H]
\begin{center}
\psfrag{xn}{$\|p_t\|$}
\psfrag{n}{$t$}
\psfrag{s}{\small{short}}
\psfrag{l}{\ \hspace{-5mm} \small{long moderate phase}}
\epsfig{width=10cm, file=bilder/orb2.eps}
\end{center}
\caption{Multi-humped orbits $(p_t)_{t\in\Z}$ exist 
  in a neighborhood of a homoclinic orbit that revisits a small neighborhood
  of the fixed point infinitely many times.\label{F4}}
\end{figure}

In non-invertible systems, similar results hold true as long as the
Jacobian $DF(p_t)$, $t\in \Z$ is invertible along all points of the
homoclinic orbit, see \cite{sw90, ka94, sa00}. This condition is
satisfied in Figure \ref{F3} since there is no intersection between
the blue line and the points of the orbit. 
Consequently, transversal intersections at
one point of the orbit result in transversal intersections at each
point (this is not true if $DF(p_t)$ is non-invertible for at
least one $t \in \Z$). 

We detect the parameter range, for
which homoclinic orbits exist. For this task, we start with the orbit,
shown in Figure \ref{F3}, and apply for fixed $\delta =
0.8$ pseudo-arclength-continuation \cite{ag90} w.r.t.\ the parameter
$\alpha$. Figure \ref{F5} displays the 
 resulting orbits $p_t(\alpha)$ over their amplitude
$$
\mathrm{amp} (p_t)_{t\in\Z} := \left(\sum_{t\in\Z} \|p_t-\xi\|^2\right)^\frac
12.
$$
Consequently, one can read off from Figure \ref{F5} that homoclinic
orbits exist for $\alpha > \bar \alpha \approx -0.046$.

\begin{figure}
\begin{center}
\psfrag{amp}{$\mathrm{amp}$}
\psfrag{l}{$\alpha$}
\epsfig{width=10cm, file=bilder/cont2.eps}
\end{center}
\caption{Continuation of homoclinic orbits.\label{F5}}
\end{figure}
 
Note that homoclinic orbits w.r.t.\ the hyperbolic fixed point $\xi$
are unstable objects that will typically not show up in simulations.
But, as we will see, this situation changes with the occurrence of a chaotic
attractor. 

\subsection{An invariant curve and its breakdown}
At the Neimark-Sacker bifurcation, see Figure \ref{F1}, an invariant
curve is born that takes over the stability from the two-periodic
orbit. For $\delta = 0.8$ and $\alpha = -0.042$ an
attracting invariant curve exists, see Figure \ref{F6}. Thus,
orbits that return to a small neighborhood of the fixed point $\xi$
cannot be observed via numerical simulation. 

\begin{figure}[H]
\begin{center}
\psfrag{x}{$x$}
\psfrag{y}{$y$}
\epsfig{width = 10cm, file=bilder/invar.eps}  
\end{center}
\caption{Parts of the stable set of $\xi$ (green) and an attracting
  invariant curve (black) for $\delta = 0.8$, $\alpha = -0.042$.\label{F6}}
\end{figure}

If we increase $\alpha$ further, the invariant curve starts to break up. 
Figure \ref{F7} shows in red a hyperbolic orbit of period $30$
together with its unstable manifold. This manifold converges towards a
stable orbit of period $30$ (black points). 

\begin{figure}[H]
\begin{center}
\psfrag{x}{$x$}
\psfrag{y}{$y$}
\epsfig{width = 12cm, file=bilder/aufbruch.eps}  
\end{center}
\caption{A hyperbolic orbit of period $30$ (red), its stable manifold
  (green) and a stable orbit of period $30$ (black) for $\delta =
  0.8$, $\alpha = -0.0397$.\label{F7}} 
\end{figure}

For this parameter setup, we also find orbits of period 16, see Figure
\ref{F8} and \ref{F9} for the corresponding bifurcation diagrams.

\begin{figure}[H]
\begin{center}
\psfrag{a}{$\alpha$}
\psfrag{p}{period}
\epsfig{width = 8cm, file=bilder/zunge1d.eps}  
\end{center}
\caption{Occurrence of periodic orbits for $\delta = 0.8$.\label{F8}} 
\end{figure}

\begin{figure}[H]
\begin{center}
\psfrag{a}{$\alpha$}
\psfrag{d}{$\delta$}
\epsfig{width = 10cm, file=bilder/zunge2.eps}  
\end{center}
\caption{Periodic orbits for various values of $\alpha$ and
  $\delta$.\label{F9}}  
\end{figure}




Finally this leads to a breakup of the invariant curve.
After this breakup, the stable set of the fixed point $0$ conquers any
neighborhood of the two-periodic orbit and thus, a heteroclinic
connection between these two objects exists, see Figure \ref{F16}.
Note that this connection is not possible in the presents of an
attracting invariant curve. 

\begin{figure}[H]
\begin{center}
\psfrag{a}{$\alpha$}
\psfrag{d}{$\delta$}
\epsfig{width = 12cm, file=bilder/het2.eps}  
\end{center}
\caption{Heteroclinic connection between the
  two-periodic orbit and 
  the fixed point $0$ w.r.t. $F^2$ for $\delta = 0.8$, $\alpha =
  -0.03$ and parts of the stable set $S_i$ for $i \in \{0,\dots,
  40\}$.\label{F16}}    
\end{figure}

The breakup of the invariant curve results in the occurrence  of a (chaotic)
attractor and the unstable set of the fixed point $\xi$
is part of this attractor. Consequently, all homoclinic
orbits lie in this attracting set as well. 
The homoclinic structure is now visible via forward iteration
and orbits approach repeatedly a small neighborhood of the fixed
point $\xi$, cf.\ Figure \ref{F15}.

\begin{figure}[H]
\begin{center}
\epsfig{width = 12cm, file=bilder/chaos.eps}  
\end{center}
\caption{Stable set (green) unstable set (red) and the orbit with
  starting point $(0.01, 0.01)$ (black) for $\delta = 0.8$, $\alpha =
  -0.038$. \label{F15}} 
\end{figure}


For the parameters $\delta = 0.8$, $\alpha = -0.038$, we compute an
orbit $(x_t)_{t \le T}$ of length $T=5\cdot 10^{10}$. A moderate
interval $[a,b]\subset \N$ is maximal with respect to $\|x_t\| \le
0.02$ for all $t \in [a,b]$. In Figure \ref{F10}, we count the number
$I(\ell)$ of moderate intervals of length $\ell$ for the trajectory
$(x_t)_{n \le T}$.  

\begin{figure}[H]
\begin{center}
\psfrag{n}{$\ell$}
\psfrag{t}{$I(\ell)$}
\epsfig{width=10cm, file=bilder/chart.eps}  
\end{center}
\caption{Number of moderate intervals of length $\ell$, dented by $I(\ell)$ 
 for an orbit of length $N=5\cdot 10^{10}$ and $\delta =
  0.8$, $\alpha = -0.038$.\label{F10}}  
\end{figure}

